\documentclass[12pt]{article}

\usepackage{scicite,times,graphicx,float,hyperref}
\usepackage[skip=0pt]{caption}
\usepackage{subcaption}
\usepackage[utf8]{inputenc}
\usepackage{enumitem}
\usepackage{booktabs}
\usepackage{multicol}
\usepackage{listings}
\usepackage{color} 


\definecolor{dkgreen}{rgb}{0,0.6,0}
\definecolor{gray}{rgb}{0.5,0.5,0.5}
\definecolor{mauve}{rgb}{0.58,0,0.82}

\lstset{
  language=Java,
  aboveskip=3mm,
  belowskip=3mm,
  showstringspaces=false,
  columns=flexible,
  basicstyle={\small\ttfamily},
  numbers=none,
  numberstyle=\tiny\color{gray},
  keywordstyle=\color{blue},
  commentstyle=\color{dkgreen},
  stringstyle=\color{mauve},
  breaklines=true,
  breakatwhitespace=true,
  tabsize=3
}

\topmargin -1.0cm
\oddsidemargin 0.0cm
\textwidth 16cm
\textheight 23cm
\footskip 1.0cm

\newenvironment{sciabstract}{%
  \begin{quote} \bf}
    {\end{quote}}

\newcounter{lastnote}
\newenvironment{scilastnote}{%
  \setcounter{lastnote}{\value{enumiv}}%
  \addtocounter{lastnote}{+1}%
  \begin{list}%
    {\arabic{lastnote}.}
    {\setlength{\leftmargin}{.22in}}
    {\setlength{\labelsep}{.5em}}
    }
    {\end{list}}

\title{Simulation of a Scientific Computation Platform\\With a Focus on Quality Attributes}

\author
{Filipe Pires [85122], João Alegria [85048]\\
  \\
  Software Architecture\\
  \normalsize{Department of Electronics, Telecommunications and Informatics}\\
  \normalsize{University of Aveiro}\\
}

\date{\today{}}

%%%%%%%%%%%%%%%%% END OF PREAMBLE %%%%%%%%%%%%%%%%%%%%%

\begin{document}

\baselineskip18pt

\maketitle

\section*{Introduction} %%%%%%%%%%%%%%%%%%%%%%%%%%%%%%%%%%%%%%%%%%%%%%%%%%%%%%%%%%%%%%%%%%%%%%%%%%%%%%%%%%%%%%%%%%%%%%%%%%%%%%%%%%%%%%%%%%%%%%%%%%%%%%%%%%%%%%%%

This report aims to describe the work developed for the third and final assignment of the course of 'Software Architecture', focused on a platform that accepts
and processes computational services requested by the scientific community.

The aim of the assignment was to design and develop a software architecture relying on four of the most relevant quality attributes:
performance, availability, scalability and usability.
Playing the role of software architects, we came up with the solution for an infrastructure for our stakeholder.
The platform here presented is capable of deploying a cluster of servers monitored by a tactic entity and whose requests from clients are distributed by a load balancer.
Although the cluster is locally simulated, the configuration is done so that it is possible to deploy in a distributed environment.

So in this report we present the architecture of our solution, justifying design decisions according to what we learned and found to be most suitable for our use case.
We also mention how the work was distributed amongst the authors.

All code developed is publicly accessible in our GitHub repository:

\url{https://github.com/FilipePires98/AS/}

\newpage
\section{Scientific Computation and Custom Systems} \label{scientificcomputation} %%%%%%%%%%%%%%%%%%%%%%%%%%%%%%%%%%%%%%%%%%%%%%%%%%%%%%%%%%%%%%%%%%%%%%%%%%%%%%

Applied computer science and mathematics often use advanced computing capabilities to understand and solve complex problems.
In practical use, computational science is typically the application of computer simulation and other forms of computation to tackle problems in various
scientific disciplines.
Scientists and engineers develop computer programs that model systems being studied and run these programs with various sets of input parameters.
In some cases, these models require massive amounts of calculations and are often executed on supercomputers or distributed computing platforms.

This project does not describe one of these models, rather it is about the development of an infrastructure capable of supporting such computationally demanding tasks.
The adopted strategy was of coordinating a cluster of processing servers.
The following tactics were selected to be implemented:
\vspace{-10pt}
\begin{itemize}[noitemsep]
  \item Computation Replicas - requests must be fairly distributed amongst the servers.
  \item Concurrency - each request runs on its own thread in a server.
  \item Redundancy - in case a server goes down, requests should be reallocated.
  \item Monitor - everything must be supervised, from the cluster's status, to the clients' identification and request treatment.
  \item Horizontal scalability - new servers can be deployed whenever necessary.
\end{itemize}
\vspace{-10pt}

\subsection{The Scenario} \label{scenario} %%%%%%%%%%%%%%%%%%%%%%

In order to test the application of the infrastructure on a scientific problem, an example scenario was adopted: the calculation of the mathematical constant $\pi$ (pi).
Being an irrational number, $\pi$ cannot be expressed as a common fraction and its decimal representation never ends and never settles into a permanently repeating pattern.
Nevertheless, in the 21st century, mathematicians and computer scientists have pursued new approaches that, when combined with increasing computational power,
extended the decimal representation of $\pi$ to many trillions of digits \cite{pisky}.
The primary motivation for these computations is as a test case to develop efficient algorithms to calculate numeric series, as well as the quest to break records.

The most commonly adopted forms of calculating $\pi$ are the iterative algorithms \cite{jorg}.
In order to only focus on the infrastructure's implementation, while simulating the actual implementation of the iterative algorithms, a control variable
corresponding to the number of iterations to be used on the calculation was introduced to regulate servers' response times.

\subsection{The Messages} \label{messages} %%%%%%%%%%%%%%%%%%%%%%

One of the fundamental constraints applied to our implementation was related to communications.
Each individual server is launched as an independent process.
This is also true for each individual client.
The orchestrator runs in an independent process as well, serving as both a load balancer and a tactic monitor - although these two are built as separate entities.
So in order for entities to communicate with each other, the TCP/IP socket technology was made a requirement.

For simplicity, only 2 service message types were defined:
\vspace{-10pt}
\begin{itemize}[noitemsep]
  \item Request: $|$ \texttt{clientID} $|$ \texttt{requestID} $|$ \texttt{01} $|$ \texttt{\# iterations} $|$
  \item Reply: $|$ \texttt{serverID} $|$ \texttt{clientID} $|$ \texttt{requestID} $|$ \texttt{02} $|$ \texttt{\# iterations} $|$ \texttt{pi} $|$
\end{itemize}
\vspace{-10pt}
In them are the following parameters:
\texttt{clientID}, a positive integer that uniquely identifies a client machine;
\texttt{serverID}, a positive integer that uniquely identifies each physical server in the cluster;
\texttt{requestID}, a positive integer that uniquely identifies a pi calculation request (computed as 1000 $\times$ \texttt{clientId} $+$ increment);
\texttt{01} and \texttt{02}, the request and reply codes;
\texttt{\# iterations}, the chosen value for the previously mentioned control variable used to simulate the number of iterations for the computation of pi
(each iteration / cycle corresponds to 1 second);
\texttt{pi}, the computed value of $\pi$. In addition, between internal server components we found the need to append to the request message some metadata to help us in implementing some mandatory features. We then extended that message format and ended up with \texttt{request-<clientHost>-<clientPort>-<request>}, a self-explanatory format but the parameters will be detailed below.

Additional message types were added during development for management purposes:
\vspace{-10pt}
\begin{itemize}[noitemsep]
  \item Health Check: \texttt{healthcheck}
  \item New Client: \texttt{newClient-<clientHost>-<clientPort>}
  \item Client Down: \texttt{clientDown-<clientId>}
  \item New Server: \texttt{newServer-<serverHost>-<serverPort>}
  \item Client Down: \texttt{serverDown-<serverId>}
  \item New Request: \texttt{newRequest-<serverId>-<request>}
  \item Request Processed: \texttt{processedRequest-<serverId>-<request>}
\end{itemize}
\vspace{-10pt}
As shown, some of these messages also need parameters:
\texttt{clientHost}, a string containing the host name of the client machine;
\texttt{clientPort}, a positive integer containing the port where the client process is running;
\texttt{serverHost}, a string containing the host name of the server machine;
\texttt{serverPort}, a positive integer containing the port where the server process is running;
\texttt{request}, a string containing the request message that triggered that event in the format mentioned above.

\newpage
\section{System Architecture} \label{architecture} %%%%%%%%%%%%%%%%%%%%%%%%%%%%%%%%%%%%%%%%%%%%%%%%%%%%%%%%%%%%%%%%%%%%%%%%%%%%%%%%%%%%%%%%%%%%%%%%%%%%%%%%%%%%%

The adopted architecture was a hybrid version of client/server and master/slave.
First of all, the clients that connect to and use our system act as a typical client entity and, from their perspective, there is only one server entity
responding to their requests.
The server, however, is in fact made up of several entities, each with its specific responsibility.
These internal entities are independent from one another and when they need to communicate they do so in a client/server manner, making the necessary requests
and awaiting for a response.

The architecture follows a master/slave model as well in the sense that the $\pi$ calculation infrastructure is comprised of a main entity (the master) and a
set of calculation servers (the slaves): the calculation servers await for their master to assign them tasks; the main entity is actually made up of 2
entities (a Load Balancer and a Tactic Manager) that run in the same process as we will detail further ahead and that does the interaction with the clients and
communicate with the slaves in order to deliver a response to the client requests.

As predefined by our stakeholder, all communications are done through web sockets.
Below, we provide a detailed description of each entity as well as of the communication mechanisms.

\subsection{Entities} \label{entities} %%%%%%%%%%%%%%%%%%%%%%%%%%

\subsubsection{Client}

The Client acts as the system entity that provides the user a graphical interface (GUI) in order to make calculation requests to the scientific computation
platform.
It is the Client that parses the user input and sends requests to the Load Balancer, awaiting for responses to arrive to provide the user with the answers he/she needs.
The user can make how many requests he/she desires, as the Client does not need to wait for each computation response in order to request another.

A key design decision was made when implementing the communication mechanism for clients.
After some deliberation, we concluded that the best approach for the web sockets' ports definition would be to make the Client responsible for choosing the port
it uses, since there could be constraints related to the machine where the process is running and it wouldn't be viable for the Client to receive the port
indication from a server that might have no knowledge about the ports already in use on the Client's machine.
This way, the Client instantiates a web socket with a port defined by itself, allowing the computation responses to be sent to it asynchronously.

A Client instantiation imposes the exchange of some control messages.
When initiating, the Client needs to send a message to notify the main server that it was created, indicating its web socket host IP address and chosen port.
In response, the main server sends to the Client an internal identifier (used for the server to recognize it in future messages).
When the Client is stopped, it sends a new message notifying the main server that it will shut down.

\subsubsection{Calculation Server}

Calculation servers hold the code for the computation of the scientific algorithms.
These entities await for requests to arrive from the main server, process them and respond directly to the clients that triggered the requests.
For our use case, the $\pi$ computation algorithm is simulated according to a parameter, as previously mentioned.
Nevertheless, the code was made so that a real algorithm implementation can quickly replace the current simulation, and allowing additional algorithms to be
inserted so that the platform supports new computation requests.

A Server is able to respond directly to the Client thanks to an internal mechanism:
the request is initially sent by the Client to the Load Balancer;
it requests some Client metadata to the Tactic Manager, adds it to the request and then the Load Balancer forwards the updated message to the Calculation Server;
with the added information, the Server gains the capacity to send the computation results to the Client without the help of the main server.
All processing of requests is parallelized, preventing any undesired bottlenecks.

Similarly to the Client, a Server needs a web socket server to receive the calculation requests, so it also needs to choose the port it will be active in.
The same physical constraints apply, so the servers need to exchange similar control messages on instantiation and shut down, including host IP address and port identification.

\subsubsection{Load Balancer} \label{lb}

We have mentioned the Load Balancer more than once, as it is a key asset to our solution, and in this section we explain why it is one of the most important
entities of our architecture.
Its existence started by being a project requirement, but we quickly understood its actual importance when handling the quality attributes we wished to guarantee.

Load Balancer's main purpose, as the name implies, is to distribute in a balanced way the incoming $\pi$ calculation requests across all registered servers.
When one or more calculation requests are sent to it, Load Balancer asks the Tactic Manager via web socket to identify the least occupied Server and assigns the
task to that Server.
It is in this stage, after the Server is chosen, that the Load Balancer asks the Tactic Manager for the Client's metadata to be inserted in the request.

It is important to state that the Load Balancer is the central entity of the infrastructure.
It is only through it that the clients are able to interact with the remainder of the infrastructure and it is only this entity that the Client has knowledge
about from the very moment it is instantiated.
The Tactic Manager only server this entity, although it holds the information about the remainder.
If the process where the Load Balancer and Tactic Manager are running dies for some reason, the platform will not work; this is not true for the Client nor the
Calculation Server.
In section \ref{constraints} we explore how the Load Balancer is able to handle availability and scalability.

\subsubsection{Tactic Manager}

If the Load Balancer is the heart of the platform, the Tactic Manager can easily be considered its brain.
It is this entity that holds the knowledge about every registered Server, every connected Client, the state of each and all, and who is processing each request
at any time.
With this knowledge, it is able to understand which Server has the least amount of pending requests (is least occupied) and provide the Load Balancer with an
informed answer.
The shared GUI of the Load Balancer and Tactic Manager is always up-to-date because of this knowledge as well.

Although the Calculation servers receive messages from the Load Balancer, they only communicate with the central server through the Tactic Manager (they only
send messages to it).
The Server initialization and shutdown control messages are sent to the Tactic Manager, that will process and store the information, also sending back the
created identifier in case of the Server creation.
Additionally, all servers when receiving or finishing the processing of a calculation request should notify this management entity of their status change.

A control mechanism already mentioned in section \ref{messages} but not yet explained is also responsibility of the Tactic Manager - the health-check of
registered servers.
This feature was added as we concluded it would be paramount when assuring some quality attributes, namely availability, performance and usability.
The flow of this process goes as follows: when the Tactic Manager is created, the Health-checking subprocess is initiated in parallel; from then on, this
subprocess sends health checks to all registered servers on a time interval established \textit{a priori} (in our scenario it was defined as 30 seconds periods);
in the event that one Server does not respond, it is considered dead and consequently deleted from the internal registries so that future requests are not sent
to it; if any requests assigned to the dead Server were incomplete, these are sent back to the Load Balancer for redistribution amongst the living Servers.

\subsection{System Components} %%%%%%%%%%%%%%%%%%%%%%%%%%%%%%%%%%

We have discussed the entities that make up the logic of the solution - these are all represented in our architecture diagram in Figure \ref{fig:pa3Arch} as well
as their interactions in Figure \ref{fig:pa3Interaction}.
However, there are additional system components that provide what is necessary for the integration of the whole infrastructure, some of them worth exploring in
depth due to their important roles.

The first two important components are the LoadDistributor, a class internal to the LoadBalancer that is responsible for the actual distribution of incoming requests,
and the HealthChecker, internal to the TacticManager with the duty to health check all registered servers and take action in case any of them does not respond.

\begin{figure}[H]
  \centering
  \begin{minipage}{\textwidth}
    \centering
    \includegraphics[width=\linewidth]{img/pa3Arch.png}
  \end{minipage}%
  \caption{Platform's architecture diagram.}
  \label{fig:pa3Arch}
\end{figure}

Then we have the SocketServer and SocketClient, which are wrapper classes for a WebSocket Server and a WebSocket Client.
They enabled us to use sockets personalized for our needs without worrying about the internal functioning of the original class, and in consequence keep our
focus on our problem domain alone.
The only consideration we needed to make was that every entity that used the SocketServer needed to pass a MessageProcessor interface instance that would supply
all of the message processing logic.
The classes that implement MessageProcessor are the Client, the Server, the LoadBalancer and the TacticManager.
An internal class called AttendUser was added to our SocketServer to handle all the connections made to the WebSocket server in parallel.

\begin{figure}[H]
  \centering
  \begin{minipage}{\textwidth}
    \centering
    \includegraphics[width=.75\linewidth]{img/pa3Interaction.png}
  \end{minipage}%
  \caption{Platform's interaction diagram.}
  \label{fig:pa3Interaction}
\end{figure}

\newpage
\subsection{User Interface} \label{ui} %%%%%%%%%%%%%%%%%%%%%%%%%% 

We defined from the very beginning that the GUIs would have to be as simple as possible, while providing both the basic and required elements for the user and
additional features that would enrich their usability and empower the user.
The resulting interfaces are presented next.

On initialization, the system parses optional arguments passed to the Java application with the help of the Commons CLI library \cite{commonscli}.
Two options are accepted as arguments: \texttt{-c} and \texttt{-s}, where the number of clients and servers (respectively) to deploy on initialization are defined.
By default, if no arguments are passed, the platform deploys 1 instance of each.

Their GUIs are presented in Figure \ref{fig:GUIs_init}.
From the screen captures taken of each UI, it is possible to conclude a few aspects.
First of all, the use of color coding helps the user immediately understand the state of the process; this is more important when the scale is bigger (more
instances of servers and/or clients).
Also, the parameters for the IP addresses and ports are pre-filled considering available ports and avoiding collisions.
Nevertheless, input fields are all active, meaning that the user can personalize them as he/she wishes to.

For inexperienced users, the title of each interface helps him/her understand the respective role, although it does not yet uniquely identify the process;
also, they can understand the purpose of each text area through the label placed above each one.
The positioning of each element (input fields, buttons, text areas) was made strategically, adopting a dashboard-like structure for the Load Balancer and
Tactic Manager GUI.

\begin{figure}[H]
  \centering
  \begin{subfigure}{.55\textwidth}
    \centering
    \includegraphics[width=.95\linewidth]{img/LBM_init.png}
    \caption{Load Balancer \& Tactic Manager GUI}
    \label{fig:LBM_init}
  \end{subfigure} \\
  \begin{subfigure}{.5\textwidth}
    \centering
    \includegraphics[width=.95\linewidth]{img/S_init.png}
    \caption{Server GUI}
    \label{fig:S_init}
  \end{subfigure}%
  \begin{subfigure}{.5\textwidth}
    \centering
    \includegraphics[width=.95\linewidth]{img/C_init.png}
    \caption{Client GUI}
    \label{fig:C_init}
  \end{subfigure}
  \caption{GUIs of each independent process in the initial state of execution.}
  \label{fig:GUIs_init}
\end{figure}

\begin{figure}[H]
  \centering
  \begin{subfigure}{.55\textwidth}
    \centering
    \includegraphics[width=.95\linewidth]{img/LBM_confirmed.png}
    \caption{Load Balancer \& Tactic Manager GUI}
    \label{fig:LBM_confirmed}
  \end{subfigure} \\
  \begin{subfigure}{.5\textwidth}
    \centering
    \includegraphics[width=.95\linewidth]{img/S_confirmed.png}
    \caption{Server GUI}
    \label{fig:S_confirmed}
  \end{subfigure}%
  \begin{subfigure}{.5\textwidth}
    \centering
    \includegraphics[width=.95\linewidth]{img/C_confirmed.png}
    \caption{Client GUI}
    \label{fig:C_confirmed}
  \end{subfigure}
  \caption{GUIs of each independent process after each is confirmed and ready to use.}
  \label{fig:GUIs_confirmed}
\end{figure}

\begin{figure}[H]
  \centering
  \begin{subfigure}{.5\textwidth}
    \centering
    \includegraphics[width=.95\linewidth]{img/S_wrong.png}
    \caption{Server GUI}
    \label{fig:S_wrong}
  \end{subfigure}%
  \begin{subfigure}{.5\textwidth}
    \centering
    \includegraphics[width=.95\linewidth]{img/C_wrong.png}
    \caption{Client GUI}
    \label{fig:C_wrong}
  \end{subfigure}
  \caption{GUIs of each independent process presenting warning messages due to overlapping ports.}
  \label{fig:GUIs_wrong}
\end{figure}

In Figure \ref{fig:GUIs_confirmed} the system is fully deployed and active.
To transition to an active (green) state, the user simply needs to confirm the values of the input fields by clicking on the "Confirm" button of each UI.
In case the input is invalid, for example if the chosen ports are already in use, a warning message in red (again, using color coding) is shown on the bottom of
the UI, as seen in Figure \ref{fig:GUIs_wrong}.
Note that, once confirmed and validated, the input fields are made inactive as they should not be changed, and that the titles of the Client and Server change
so that they can be uniquely identified.

\clearpage
\begin{figure}[H]
  \centering
  \begin{subfigure}{.55\textwidth}
    \centering
    \includegraphics[width=.95\linewidth]{img/LBM_active.png}
    \caption{Load Balancer \& Tactic Manager GUI}
    \label{fig:LBM_active}
  \end{subfigure} \\
  \begin{subfigure}{.5\textwidth}
    \centering
    \includegraphics[width=.95\linewidth]{img/S_active.png}
    \caption{Server GUI}
    \label{fig:S_active}
  \end{subfigure}%
  \begin{subfigure}{.5\textwidth}
    \centering
    \includegraphics[width=.95\linewidth]{img/C_active.png}
    \caption{Client GUI}
    \label{fig:C_active}
  \end{subfigure}
  \caption{GUIs of each independent process during the processing of multiple requests.}
  \label{fig:GUIs_active}
\end{figure}

With the input fields validated, the system can start receiving requests and processing them.
In Figure \ref{fig:GUIs_active}, we present an example usage of the infrastructure, where several requests are done simultaneously, with different values for the
number of pi iterations, by different clients and distributed amongst different servers.

The Server GUI is merely informative.
It contains the Server status, its own IP address and port and of the main server, and 2 lists, one containing the received requests and the other those
already processed.
The interface has an additional "Stop" button, meant to safely shutdown the Server through control messages.

The Client GUI is similar, but instead of 2, it contains 3 lists: one for the requests sent, one for those under processing and one for those already
executed by the infrastructure and containing the results of the pi calculations.
Besides the "Stop" button, this interface also has a "Send" button, obviously.
The parameter for the number of iterations for the calculation of pi is placed right beside this button and the interaction method is through a spinner in order
to avoid user errors.

The Load Balancer and Tactic Manager GUI naturally holds the most information about the infrastructure.
Besides presenting the IP addresses and ports of the Load Balancer and of the Tactic Manager, it shows 4 lists, each with a specific purpose.
The top 2 lists contain the requests successfully processed (on the left) and the requests being processed (on the right), both with the ID of the Server
responsible for each request.
The bottom 2 lists contain the connected servers (on the left) and the connected clients (on the right).
The servers are accompanied with their current occupation (the number of requests they are processing).
On the left of these lists lies a column with global information and operations; here it is possible to learn the total number of requests processed and of
those being processed and the total occupation of the infrastructure; the user can also deploy new clients or servers through the "Add New Client" and "Add New
Server" buttons; and he/she can also shutdown the process through the "Stop" button.

\begin{figure}[H]
  \centering
  \begin{minipage}{\textwidth}
    \centering
    \includegraphics[width=\linewidth]{img/LBM_large.png}
  \end{minipage}%
  \caption{Load Balancer and Tactic Manager's GUI in expanded view during the processing of multiple requests in a cluster of 20 Calculation servers.}
  \label{fig:LBM_large}
\end{figure}

Finally we present a screen capture of the Load Balancer and Tactic Manager's GUI in expanded view (1920x1080) to show how the interface is able to adapt to
different sizes without loosing its natural format.
In the scenario of Figure \ref{fig:LBM_large}, a cluster of 20 servers is deployed and the infrastructure is tested with many clients.

\newpage
\section{Architecture Constraints} \label{constraints} %%%%%%%%%%%%%%%%%%%%%%%%%%%%%%%%%%%%%%%%%%%%%%%%%%%%%%%%%%%%%%%%%%%%%%%%%%%%%%%%%%%%%%%%%%%%%%%%%%%%%%%%%

We have stated that the designed architecture would rely on four of the most relevant quality attributes - performance, availability, scalability and usability -
and throughout this document these are visible in numerous circumstances.
However, a critical analysis to the implementation was performed in order to determine how are these attributes actually ingrained and how well are they assured.

\subsection{Performance} \label{performance}

As detailed by our stakeholder, this quality attribute should focus on the following points:
computational replicas, providing multiple instances able to calculate $\pi$, which translates to faster response times, as requests can be processed in
parallel assuming that they are equally distributed amongst replicas;
concurrency, defining by design that each request is processed in its own thread and providing synchronization with resort to Reentrant Locks.

With this in mind, we developed several mechanisms to ensure that the the guidelines were followed and the solution was fit for its purposes.
To illustrate the load distribution amongst replicas, we present a code snippet from the LoadDistributer:

\begin{lstlisting}[label={loadDistributor}]
//LoadDistributor
String[] server=myClient.send("leastOccupiedServer").split("-");
if(server[1].equals("none")){
  // code hidden (not relevant here)
}
else{
  SocketClient clientSocket=new SocketClient(
    // code hidden (not relevant here)
  );
  try {
    for(String msg:messages){
      String[] tmp=msg.replaceAll("\\s+","").split("\\|");
      String[] processed=Arrays.copyOfRange(tmp,1,tmp.length);
      String[] client=myClient.send("clientInfo-"+processed[0]).split("-");
      clientSocket.send("request-" + client[2] +"-"+ client[3] +"-"+ msg);
    }
  } catch (IOException ex) {
    myClient.send("serverDown-"+server[1]);
  }
  clientSocket.close();
}
\end{lstlisting}

\newpage
In relation to the concurrency aspect and the synchronization, in this project we made sure that all clients were handled by separate processes and every request
made by each client also treated in a dedicated thread on a Server.
Reentrant Lock synchronization was only found useful in the Tactic Manager; as it stores information that must be at all times consistent and reliable, we
implemented an internal monitor using a \textit{ReentrantLock} instance to ensure the mutually exclusive access to the global information between all threads
created to handle each message arriving at the Tactic Manager.

3 code snippets are presented below, as they were found relevant for this quality attribute assurance.
The first belongs to the SocketServer wrapper, showing that each client is handled by an independent thread;
the second belongs to the Calculation Server, showing that each incoming request is handled by an independent thread;
and the last belongs to the Tactic Manager's internal monitor.

\begin{lstlisting}
//SocketServer
socket = new ServerSocket(this.port);
mp.setSocketStatus(1);
while(continueRunning){
  Socket inSocket = socket.accept();
  Thread t=new Thread(new AttendClient(inSocket));
  t.start();
}
socket.close();
\end{lstlisting}

\begin{lstlisting}
//Server
String[] processedMessage = message.split("-");
switch(processedMessage[0]){
  // code hidden (not relevant here)
  case "request":
    SocketClient manager = new SocketClient(tmHost,tmPort);
    try {
      manager.send("newRequest-"+this.id+"-"+processedMessage[3]);
      manager.close();
      processingRequests.add(processedMessage[3]);updateProcessing();
      PiCalculation request = new PiCalculation(processedMessage[1], Integer.valueOf(processedMessage[2]), processedMessage[3]);
      Thread requestProcessing = new Thread(request);
      requestProcessing.start();
    } catch (IOException ex) {
      Logger.getLogger(Server.class.getName()).log(Level.SEVERE, null, ex);
    }
    break;
  }
}
\end{lstlisting}

\begin{lstlisting}
//ClusterInfo(Monitor)
private final ReentrantLock rl;
// code hidden (not relevant here)
public void addServer(String host, int port){
  rl.lock();
  int id=getNewServerId();
  ServerInfo si = new ServerInfo(id, host, port);
  serverInfo.put(id, si);
  SocketClient client=new SocketClient(host, port);
  try {client.send("serverId-"+id);} 
  catch (IOException ex) {removeServer(id);}
  client.close();updateServers();
  rl.unlock();
}
\end{lstlisting}

\subsection{Availability} \label{availability}

The stakeholder's guidelines outlined 2 key points related to availability:
the first was on providing redundancy, meaning that on the event of a Server going down, the system should react and reallocate every pending request that Server
was responsible for to other servers;
the second was on the implementation of a monitoring entity, enabling a supervision of the entire cluster status, active servers, registered clients and requests status.

Redundancy could be achieved by one of 2 alternatives:
either the task distribution would have associated a message receiving acknowledgement from the servers, that in the event it did not arrive, it would mean that
the Server is down;
or servers would have to be consistently monitored to understand their status.
In our solution, we actually adopted both strategies, resorting to the already discussed health check mechanism for the later.
Going back to the code snippet from the LoadDistributor, it is possible to understand that when a exception is raised it means that the Server did not reply
with an acknowledge message, so it is assumed dead, removed from the internal registry and the request reallocation process is executed.
The code snippet below belongs to ClusterInfo's internal class HealthCheck and presents the logic behind this mechanism.
The flow of this process is described in our interaction diagram as well (see Figure \ref{fig:pa3Interaction}).
All these mechanisms and the constant (but consistent) updates to Tactic Manager's registry ensure that the infrastructure is available to users and that they
see a faithful representation of the system at all times.

\begin{lstlisting}
//HealthCheck
while(true){
  try {Thread.sleep(30000);}
  catch (InterruptedException ex) {
    Logger.getLogger(TacticManager.class.getName()).log(Level.SEVERE, null, ex);}
  for(ServerInfo si : serverInfo.values()){
    try {
      SocketClient client=new SocketClient(si.getHost(), si.getPort());
      client.send("healthcheck");client.close();
    } catch (IOException ex) {removeServer(si.getId());}
  }
}
\end{lstlisting}

\subsection{Scalability} \label{scalability}

Our stakeholder also wished for the platform to be scalable.
The way we tackled this was to allow for the user controlling the system to scale the infrastructure horizontally according to its needs.
By having access to the information of the Load Balancer and Tactic Manager GUI, the infrastructure manager can manually determine when it is necessary to deploy
additional Calculation servers and deploy them without needing to worry about technical aspects such as choosing the Server's port.

On the code snippet below belonging to the LoadBalancerTacticManager class, the process of deploying a new Server is depicted.
The Server receives an automatically generated unique identifier from the Tactic Manager and is able to provide a suggestion for the port to be used by the new
host - this value assumes that the user resorted to the recommended ports on previous servers and has no consideration for external services that might be using
the suggested port.
The independent process is launched with the help of our Utilities class.

\begin{lstlisting}
//LoadBalancerTacticManager
int serverID = tm.getNewServerID();
String userDir = System.getProperty("user.dir");
String jars = userDir + "/dist/lib/*:";
String[] commands = new String[1];
commands[0] = "java -cp " + jars + userDir + "/build/classes entities.Server "  + (6100+serverID+1) + " " + this.hostTM.getText() + " " + Integer.valueOf(this.portTM.getText()); // launch Servers
try {
    Utilities.runProcess(commands);
} catch (Exception ex) {
    System.err.println("Error: unable to assign process to an entity.");
    ex.printStackTrace();
}
\end{lstlisting}

\subsection{Usability} \label{usability}
% All GUI must be driven by Usability for the EVALUATION PROCESS: usage, trace, verification,
% validation, etc. You can resort to the tactics and anything else you think is necessary. The
% stakeholder will use the GUI to validate your implementation.

It is only natural to expect that the interface between the user and the system is intuitive and allows performing the required actions for the proper usage of
the system.
But we were aware that our stakeholder would evaluate our solution based on the graphical interfaces we provided, so this quality attribute had also a great
weight on our development strategy.

We focused on the 5 quality components intrinsic to usability: learnability, efficiency, memorability, errors and satisfaction.
Starting with learnability, we made sure the GUIs were self explanatory by placing labels and suggestive terms on each interaction element.
Also, for information entries out of the format predefined by the stakeholder, we used actual words instead of codification (e.g. the "Connected Servers" list,
where entries follow the format \texttt{Occupation:X|Server:Y|Host:Z|Port:K}).

The UIs were made efficient in the sense that for every action performed, the user receives an immediate reaction, even if his/her request has not been completed yet.
This means that, once users are comfortable with the interface, their workflow can be very quick as they do not waist time determining whether their action was
captured or not.
At this front, the color codification also comes in hand, as it allows for an experienced user with access to multiple Server GUIs to almost instantly detect
if a Server is down.

With regards to memorability, what we did was to resort to well known interaction methods like buttons and spinners.
For the Load Balancer and Tactic Manager GUI the adopted dashboard-like structure might have a positive influence on people used to deal with such UIs.

User error handling was made as tight as possible, preventing the user to make mistakes that could compromise the entire platform or parts of it.
The spinner placed on the Client GUI for selecting the number of iterations is a good example of this.
However, it is worth mentioning that the IP address and port input fields all assume that the user fills them correctly; we did not insist on protecting these
input fields 100\% as if a user does not know how to deal them, perhaps it is not yet qualified to be working with such infrastructure.
It is also assumed that the user confirms each GUI in the correct order: first the LoadBalancerTacticManager, then the servers and only then the clients; in
a real case scenario this is the order in which each entity would be deployed, so we found it to be something not worth worrying about.

Finally with regards to user satisfaction, the interface simplicity and the usage of small details previously mentioned like labels and colors made us confident
that the quality provided is near the desired for such scenario.
We kept the healtcheck messages hidden as they do not provide any value to the user and they serve their purpose being transparent to him/her.
Additionally, by exposing colleagues from other departments to the system (with a brief introduction) and asking for them to evaluate in terms of user experience,
they showed positive results with minor tweaks to make.

\newpage
\section{Additional Remarks} \label{remarks} %%%%%%%%%%%%%%%%%%%%%%%%%%%%%%%%%%%%%%%%%%%%%%%%%%%%%%%%%%%%%%%%%%%%%%%%%%%%%%%%%%%%%%%%%%%%%%%%%%%%%%%%%%%%%%%%%%%

\subsection{Documentation} \label{documentation} %%%%%%%%%%%%%%%%

We understand that a solution of this kind is generic in the sense that it can be applied to other scenarios other than the scientific one where high computation
is required, without much need of refactoring.
In the event this would ever occur and that we would not be the development team responsible for the infrastructure adaptation to the new context,
the existence of code documentation would clearly be a valuable asset to hold.

With this in mind, and also as a tool to help us understand the code that the other team member had developed during the implementation phase, we applied efforts
on the writting of Javadoc for the Java classes and methods.
This and other concerns such as code readability, code style consistency and intuitive variable and function naming conventions are a good habit we try to maintain
in all of our software development assignments.

\subsection{Assignment Contributions} \label{contributions} %%%%%

As with the second assignment, the entire development phase took place in a time where on-site cooperation was not possible, so we resorted to online
communication platforms to debate decisions and discuss difficulties.
Team scheduling allowed us to work on the project simultaneously, so no member suffered from unbalanced workloads.
The dimension of the project did not appeal to the usage of repository pull requests and other synchronization tools.
However, each small solution was verified and agreed by both team members.

Having said this, the influence of both team members is present in all components, but one might say that each had stronger responsibilities on a set of project aspects:
Filipe took care of the execution of the individual Java processes, while João developed the classes related to the main server;
Filipe then focused on the servers, while João did the clients;
João was also responsible for the communication channels, while Filipe worried about the GUIs;
Filipe made sure everything was coherent throughout the report and the code
documentation, while João did the class and interaction diagrams and solved the most critical issues regarding the configuration and synchronization of the cluster.
In terms of work percentage, we believe it was about 50\% for each student.

\newpage
\section*{Conclusions} \label{conclusions} %%%%%%%%%%%%%%%%%%%%%%%%%%%%%%%%%%%%%%%%%%%%%%%%%%%%%%%%%%%%%%%%%%%%%%%%%%%%%%%%%%%%%%%%%%%%%%%%%%%%%%%%%%%%%%%%%%%%%

For this software development assignment, our stakeholder proposed to us the design and implementation of a platform made up of a cluster of servers to provide
clients with a computation infrastructure.
With a well defined target audience and a list of constraints in terms of features to provide through the user interfaces, the stakeholder moved on to define
a set of quality attributes on which the software architecture should rely.

After a planning phase, we executed the implementation under the time interval defined and added some features beyond the requested in search for pleasing our client.
These quality attributes - performance, availability, scalability and usability - were our main focus from the beginning and it is our belief that we were able
to achieve high quality results considering the requirements.
Our infrastructure is easily controlled through the 3 different UIs, supports horizontal scaling by manually adding servers through the click of a button,
contains mechanisms for reallocating requests in case of server failures remaining constantly available (as long as the main server and at least one calculation
server are still up and running), and provides fast, asynchronous and fairly distributed execution of client requests for the computation of $\pi$.

Our overall perspective on the results is very positive and we believe that the solution can indeed be adapted both to other scenarios and even real scenarios
with little code modifications.
For future work we would add a feature for shutting down servers on the main server and then focus on security-related quality attributes.

\begin{thebibliography}{9} %%%%%%%%%%%%%%%%%%%%%%%%%%%%%%%%%%%%%%%%%%%%%%%%%%%%%%%%%%%%%%%%%%%%%%%%%%%%%%%%%%%%%%%%%%%%%%%%%%%%%%%%%%%%%%%%%%%%%%%%%%%%%%%%%%%%%
  \bibliographystyle{Science}

  \bibitem{pisky}
  \textit{Pi in the sky: Calculating a record-breaking 31.4 trillion digits of Archimedes' constant on Google Cloud}.
  Taken from \href{https://web.archive.org/web/20191019023120/https://cloud.google.com/blog/products/compute/calculating-31-4-trillion-digits-of-archimedes-constant-on-google-cloud}{web.archive.org}.
  Retrieved May 2020.

  \bibitem{jorg}
  Arndt, Jorg \& Haenel, Christoph (2006).
  \textit{Pi Unleashed}, in Springer-Verlag.
  ISBN 978-3-540-66572-4.
  Retrieved May 2020.
  English translation by Catriona and David Lischka.

  \bibitem{commonscli}
  The Apache Software Foundation,
  \textit{Commons CLI}.
  \href{https://commons.apache.org/proper/commons-cli/}{commons.apache.org}.
  Retrieved May 2020.

  % \bibitem{uml}
  %   Object Management Group,
  %   \textit{What is UML},
  %   \url{https://www.uml.org/what-is-uml.htm},
  %   accessed in March 2020.

\end{thebibliography}

\clearpage

\end{document}




















